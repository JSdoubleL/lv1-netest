\documentclass[11pt]{article}
\usepackage[utf8]{inputenc}
\usepackage[top=1in, bottom=1in, left=1in, right=1in]{geometry}
\usepackage{setspace}
\usepackage{parskip}
\setstretch{1.25}
% \usepackage[numbers,sort&compress]{natbib}
% \bibliographystyle{unsrtnat}
\usepackage{amsthm}
\usepackage{amsmath}
\usepackage[normalem]{ulem}

\newtheorem{theorem}{Theorem}[section]
\newtheorem{lemma}[theorem]{Lemma}
\newtheorem*{remark}{Remark}
     
\title{Implementation Write Up}
\author{James Willson}

\begin{document}

\maketitle
\section{Progress}
\begin{itemize}
	\item Software takes input that is a Nexus file (the same input format that is used by PAUP*), using 0 and 1 character states. 
	\item It outputs a tree, where for each cycle the removed taxa is attached to one of the two attachment points. In other words, it returns one of the trees in the network.
	\item I've tested it on some basic perfect test data, such as a single cycle, two cycles attached by an edge, etc., and I haven't found any obvious bugs. 
\end{itemize}

\section{Limitations}
\begin{itemize}
	\item I have not implemented any way to handle four-degree polytomies (PAUP* cannot calculate a tree from only three taxa of course). I am not exactly sure what to do in this case.
	\item I have not added the ability to read 4-state DNA characters. This should not be hard to do, if all we want to do is filter out sites containing more than 2 character states.
	\item Also, since the package I was using to handle trees does not really support networks, the software currently does not output the full network. However, it should not be too difficult to implement something to handle this myself.
\end{itemize}

\end{document}
